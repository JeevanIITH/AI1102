\documentclass[journal,12pt,twocolumn]{IEEEtran}

\usepackage{setspace}
\usepackage{gensymb}
\singlespacing
\usepackage[cmex10]{amsmath}
\usepackage{amssymb}
\usepackage{xurl}
\usepackage{tabularx}
\usepackage{amsthm}
\usepackage{comment}
\usepackage{mathrsfs}
\usepackage{txfonts}
\usepackage{stfloats}
\usepackage{bm}
\usepackage{cite}
\usepackage{cases}
\usepackage{subfig}
\usepackage{amsmath}
\usepackage{longtable}
\usepackage{multirow}
\usepackage{multicol}

\usepackage{enumitem}
\usepackage{mathtools}
\usepackage{steinmetz}
\usepackage{tikz}
\usepackage{circuitikz}
\usepackage{verbatim}
\usepackage{tfrupee}
\usepackage[breaklinks=true]{hyperref}
\usepackage{graphicx}
\usepackage{tkz-euclide}

\usetikzlibrary{calc,math}
\usepackage{listings}
    \usepackage{color}                                            %%
    \usepackage{array}                                            %%
    \usepackage{longtable}                                        %%
    \usepackage{calc}                                             %%
    \usepackage{multirow}                                         %%
    \usepackage{hhline}                                           %%
    \usepackage{ifthen}                                           %%
    \usepackage{lscape}     
\usepackage{multicol}
\usepackage{chngcntr}

\DeclareMathOperator*{\Res}{Res}

\renewcommand\thesection{\arabic{section}}
\renewcommand\thesubsection{\thesection.\arabic{subsection}}
\renewcommand\thesubsubsection{\thesubsection.\arabic{subsubsection}}

\renewcommand\thesectiondis{\arabic{section}}
\renewcommand\thesubsectiondis{\thesectiondis.\arabic{subsection}}
\renewcommand\thesubsubsectiondis{\thesubsectiondis.\arabic{subsubsection}}


\hyphenation{op-tical net-works semi-conduc-tor}
\def\inputGnumericTable{}                                 %%

\lstset{
%language=C,
frame=single, 
breaklines=true,
columns=fullflexible
}
\begin{document}


\newtheorem{theorem}{Theorem}[section]
\newtheorem{problem}{Problem}
\newtheorem{proposition}{Proposition}[section]
\newtheorem{lemma}{Lemma}[section]
\newtheorem{corollary}[theorem]{Corollary}
\newtheorem{example}{Example}[section]
\newtheorem{definition}[problem]{Definition}

\newcommand{\BEQA}{\begin{eqnarray}}
\newcommand{\EEQA}{\end{eqnarray}}
\newcommand{\define}{\stackrel{\triangle}{=}}
\bibliographystyle{IEEEtran}
\raggedbottom
\setlength{\parindent}{0pt}
\providecommand{\mbf}{\mathbf}
\providecommand{\pr}[1]{\ensuremath{\Pr\left(#1\right)}}
\providecommand{\qfunc}[1]{\ensuremath{Q\left(#1\right)}}
\providecommand{\sbrak}[1]{\ensuremath{{}\left[#1\right]}}
\providecommand{\lsbrak}[1]{\ensuremath{{}\left[#1\right.}}
\providecommand{\rsbrak}[1]{\ensuremath{{}\left.#1\right]}}
\providecommand{\brak}[1]{\ensuremath{\left(#1\right)}}
\providecommand{\lbrak}[1]{\ensuremath{\left(#1\right.}}
\providecommand{\rbrak}[1]{\ensuremath{\left.#1\right)}}
\providecommand{\cbrak}[1]{\ensuremath{\left\{#1\right\}}}
\providecommand{\lcbrak}[1]{\ensuremath{\left\{#1\right.}}
\providecommand{\rcbrak}[1]{\ensuremath{\left.#1\right\}}}
\theoremstyle{remark}
\newtheorem{rem}{Remark}
\newcommand{\sgn}{\mathop{\mathrm{sgn}}}
\providecommand{\abs}[1]{\vert#1\vert}
\providecommand{\res}[1]{\Res\displaylimits_{#1}} 
\providecommand{\norm}[1]{\lVert#1\rVert}
%\providecommand{\norm}[1]{\lVert#1\rVert}
\providecommand{\mtx}[1]{\mathbf{#1}}
\providecommand{\mean}[1]{E[ #1 ]}
\providecommand{\fourier}{\overset{\mathcal{F}}{ \rightleftharpoons}}
%\providecommand{\hilbert}{\overset{\mathcal{H}}{ \rightleftharpoons}}
\providecommand{\system}{\overset{\mathcal{H}}{ \longleftrightarrow}}
	%\newcommand{\solution}[2]{\textbf{Solution:}{#1}}
\newcommand{\solution}{\noindent \textbf{Solution: }}
\newcommand{\cosec}{\,\text{cosec}\,}
\providecommand{\dec}[2]{\ensuremath{\overset{#1}{\underset{#2}{\gtrless}}}}
\newcommand{\myvec}[1]{\ensuremath{\begin{pmatrix}#1\end{pmatrix}}}
\newcommand{\mydet}[1]{\ensuremath{\begin{vmatrix}#1\end{vmatrix}}}
\newcommand*{\permcomb}[4][0mu]{{{}^{#3}\mkern#1#2_{#4}}}
\newcommand*{\perm}[1][-3mu]{\permcomb[#1]{P}}
\newcommand*{\comb}[1][-1mu]{\permcomb[#1]{C}}
\numberwithin{equation}{subsection}
\makeatletter
\@addtoreset{figure}{problem}
\makeatother
\let\StandardTheFigure\thefigure
\let\vec\mathbf
\renewcommand{\thefigure}{\theproblem}
\def\putbox#1#2#3{\makebox[0in][l]{\makebox[#1][l]{}\raisebox{\baselineskip}[0in][0in]{\raisebox{#2}[0in][0in]{#3}}}}
     \def\rightbox#1{\makebox[0in][r]{#1}}
     \def\centbox#1{\makebox[0in]{#1}}
     \def\topbox#1{\raisebox{-\baselineskip}[0in][0in]{#1}}
     \def\midbox#1{\raisebox{-0.5\baselineskip}[0in][0in]{#1}}
\vspace{3cm}
\title{Assignment 3}
\author{CS20BTECH11047}
\maketitle
\newpage
\bigskip
\renewcommand{\thefigure}{\arabic{figure}}
\renewcommand{\thetable}{\arabic{table}}
Download all python codes from 
\begin{lstlisting}
https://github.com/JeevanIITH/AI1102/blob/main/
assignment3/assignment3.py
\end{lstlisting}
%
and latex codes from 
%
\begin{lstlisting}
https://github.com/JeevanIITH/AI1102/blob/main/
assignment3/assignment3.tex
\end{lstlisting}
\section*{GATE EC 2012 Q.1}
Two independent random variables $X$ and $Y$ are uniformly distributed in the interval [-1,1]. The probability that max $[X,Y]$ is less than $\dfrac{1}{2}$ is 
\begin{multicols}{4}
\begin{enumerate}
\item $\dfrac{3}{4}$
\item $\dfrac{9}{16}$
\item $\dfrac{1}{4}$
\item $\dfrac{2}{3}$
\end{enumerate}
\end{multicols}

\section*{Solution}
 Since the random variable $X$ is uniformly distributed in the interval [-1,1],let $f_{X}(x)=k$.Then
 \begin{align}
 \int_{-1}^{1}f_{X}(x) dx = \int_{-1}^{1} k dx=1\\
 \Rightarrow k=\dfrac{1}{2} \\
 \intertext{So}  f_{X}(x) = \dfrac{1}{2}  \quad ,  x \in (-1,1)\\
 \intertext{Similarly}  f_{Y}(y) = \dfrac{1}{2}  \quad ,  y \in (-1,1)
 \end{align}
Now
 \begin{align}
 max[X,Y] < \dfrac{1}{2} \Rightarrow  \left(X<\dfrac{1}{2}\right) . \left(Y<\dfrac{1}{2}\right)\\
 \intertext{So} P\left(max[X,Y] < \dfrac{1}{2}\right)= P\left(\left(X<\dfrac{1}{2}\right) . \left(Y<\dfrac{1}{2}\right)  \right)
 \end{align}
 Since $X$ and $Y$ are independent 
 \begin{align}
 P\left(\left(X<\dfrac{1}{2}\right) . \left(Y<\dfrac{1}{2}\right)  \right) = P \left(X<\dfrac{1}{2} \right).  P \left(Y<\dfrac{1}{2} \right)
 \end{align}
 Now 
 \begin{align}
 P\left( X<\dfrac{1}{2} \right)= \int _{-1}^{\frac{1}{2}} f_{X}(x) dx\\
 = \int _{-1}^{\frac{1}{2}} \dfrac{1}{2} dx= \dfrac{1}{2} \left( \dfrac{1}{2}-(-1) \right) =\dfrac{3}{4}
 \intertext{similarly}  P\left( Y<\dfrac{1}{2} \right) =\dfrac{3}{4}
 \end{align}
Using (1.0.10) and (1.0.9) in (1.0.7), we get
\begin{align}
P\left(max[X,Y] < \dfrac{1}{2}\right)= \dfrac{3}{4} . \dfrac{3}{4}= \dfrac{9}{16}
\end{align} 
So option 2 is correct answer 
\section*{Alternative solution}
\begin{align}
P\left(max[X,Y] < \dfrac{1}{2}\right)=P \left(X>Y, X<\dfrac{1}{2} \right)+ P\left( Y>X, Y<\dfrac{1}{2}\right)
\end{align}
Now
\begin{align}
P \left(X>Y, X<\dfrac{1}{2} \right) = \sum_{x=-1}^{1/2} P\left(Y<x\right). P(X=x)\\
\intertext{using eq (0.0.8)}
\sum_{x=-1}^{1/2} P\left(Y<x\right). P(X=x)=\int_{-1}^{\frac{1}{2}} \left( \int_{-1}^{x} \dfrac{1}{2} dy \right). \dfrac{1}{2} dx\\
= \int_{-1}^{\frac{1}{2}} \left( \dfrac{1}{2}(x+1) \right). \dfrac{1}{2} dx\\
=\dfrac{1}{4}  \left( \dfrac{x^2}{2} + x \right) \big|_{-1}^{1/2}= \dfrac{9}{32}
\end{align}
So 
\begin{align}
P \left(X>Y, X<\dfrac{1}{2} \right)= \dfrac{9}{32}\\
\intertext{similarly} P\left( Y>X, Y<\dfrac{1}{2}\right)=\dfrac{9}{32}
\end{align}
Putting (0.0.17) ,(0.0.18) in (0.0.13) we get
\begin{align}
P\left(max[X,Y] < \dfrac{1}{2}\right)= \dfrac{9}{32} + \dfrac{9}{32}=\dfrac{9}{16}
\end{align}













\end{document}
